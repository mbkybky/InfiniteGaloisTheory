% In this file you should put the actual content of the blueprint.
% It will be used both by the web and the print version.
% It should *not* include the \begin{document}
%
% If you want to split the blueprint content into several files then
% the current file can be a simple sequence of \input. Otherwise It
% can start with a \section or \chapter for instance.

\chapter{Mising Lemma in Mathlib}

\section{Galois}

\begin{lemma}
	\lean{IntermediateField.subgroup_equiv_aut}
	\leanok
	If $H$ is a subgroup of $Gal(L/K)$, then $Gal(L / fixedField H)$ is isomorphic to $H$.
\end{lemma}

\begin{lemma}
	\lean{AlgHom.toAlgEquiv_fieldRange}
	\leanok
	Let $ K, L, E $ be fields and $ L $ and $ E $ are $ K $-algebras, $\sigma : L \to E $ is an $ K $-algebra homomorphism, them there is a $ K $-algebra isomomorphism between $ L $ and $ E $.
\end{lemma}

%\begin{lemma}
%	\lean{}
%	\leanok
%\end{lemma}



\chapter{Profinite Group}

\begin{definition}
	\lean{ProfiniteGrp}
	\leanok
	A profinite group is a topological group that is compact and totally disconnected.
\end{definition}

\begin{lemma}
	If $G$ is a profinite topological group, $H$ is a topological group that is isomorphic to $G$, then $G$ is profinite.
\end{lemma}

\begin{lemma}
	Product of (arbitrarily many) profinite groups is profinite.
\end{lemma}

\begin{lemma}
	If $G$ is a profinite topological group, then every neighborhood of the identity contains an open normal subgroup.
\end{lemma}

\begin{theorem}
	If $G$ is a profinite topological group, $H$ is a subgroup of $G$, then the following three statements
	are equivalent.
	\begin{itemize}
		\item[(i)] $ H $ is closed.
		\item[(ii)] $ H $ is profinite.
		\item[(iii)] $ H $ is the intersection of a family of open subgroups.
	\end{itemize}
\end{theorem}

\begin{lemma}
	If $G$ is a profinite topological group, $H$ is a closed subgroup of $G$, then the homogeneous space $G / H$ is compact and totally disconnected. Moreover, if $H$ is normal, then $G / H$ is a profinite topological group.
\end{lemma}

\begin{lemma}
	A subgroup of a profinite topological group is open if and only if it is closed and has finite index.
\end{lemma}

\begin{theorem}
	A topological group is profinite if and only if it is the projective limit of finite groups, each given the discrete topology.
\end{theorem}



\chapter{Infinite Galois theory}
\label{section-infinite-galois}

\noindent
The Galois group comes with a canonical topology.

\begin{lemma}\cite[\href{https://stacks.math.columbia.edu/tag/0BMJ}{Lemma 0BMJ}]{stacks-project}
	\label{lemma-galois-profinite}
	Let $E/F$ be a Galois extension. Endow $\text{Gal}(E/F)$ with the coarsest
	topology such that
	$$
	\text{Gal}(E/F) \times E \longrightarrow E
	$$
	is continuous when $E$ is given the discrete topology. Then
	\begin{enumerate}
		\item for any topological space $X$ and map $X \to \text{Aut}(E/F)$
		such that the action $X \times E \to E$ is continuous the induced map
		$X \to \text{Gal}(E/F)$ is continuous,
		\item this topology turns $\text{Gal}(E/F)$ into
		a profinite topological group.
	\end{enumerate}
\end{lemma}



\begin{lemma}\cite[\href{https://stacks.math.columbia.edu/tag/0BMK}{Lemma 0BMK}]{stacks-project}
	\label{lemma-galois-infinite}
	Let $L/M/K$ be a tower of fields. Assume both $L/K$ and
	$M/K$ are Galois. Then there is a canonical surjective continuous
	homomorphism $c : \text{Gal}(L/K) \to \text{Gal}(M/K)$.
\end{lemma}



\noindent
Here is a more standard way to think about
the Galois group of an infinite Galois extension.

\begin{lemma}\cite[\href{https://stacks.math.columbia.edu/tag/0BU2}{Lemma 0BU2}]{stacks-project}
	\label{lemma-infinite-galois-limit}
	Let $L/K$ be a Galois extension with Galois group $G$.
	Let $\Lambda$ be the set of finite Galois subextensions,
	i.e., $\lambda \in \Lambda$ corresponds to $L/L_\lambda/K$
	with $L_\lambda/K$ finite Galois with Galois group $G_\lambda$.
	Define a partial ordering on $\Lambda$ by the rule
	$\lambda \geq \lambda'$ if and only if
	$L_\lambda \supset L_{\lambda'}$. Then
	\begin{enumerate}
		\item $\Lambda$ is a directed partially ordered set,
		\item $L_\lambda$ is a system of $K$-extensions over $\Lambda$
		and $L = \colim L_\lambda$,
		\item $G_\lambda$ is an inverse system of finite groups over $\Lambda$,
		the transition maps are surjective, and
		$$
		G = \lim_{\lambda \in \Lambda} G_\lambda
		$$
		as a profinite group, and
		\item each of the projections $G \to G_\lambda$ is continuous and surjective.
	\end{enumerate}
\end{lemma}

\begin{theorem}[Fundamental theorem of infinite Galois theory]\cite[\href{https://stacks.math.columbia.edu/tag/0BML}{Theorem 0BML}]{stacks-project}
	\label{theorem-inifinite-galois-theory}
	Let $L/K$ be a Galois extension. Let $G = \text{Gal}(L/K)$
	be the Galois group viewed as a profinite topological group
	(Lemma \ref{lemma-galois-profinite}). Then we have $K = L^G$ and the map
	$$
	\{\text{closed subgroups of }G\}
	\longrightarrow
	\{\text{subextensions }L/M/K\},\quad
	H \longmapsto L^H
	$$
	is a bijection whose inverse maps $M$ to $\text{Gal}(L/M)$.
	The finite subextensions $M$ correspond exactly to the open
	subgroups $H \subset G$. The normal closed subgroups $H$ of $G$
	correspond exactly to subextensions $M$ Galois over $K$.
\end{theorem}

\begin{lemma}\cite[\href{https://stacks.math.columbia.edu/tag/0BMM}{Lemma 0BMM}]{stacks-project}
	\label{lemma-ses-infinite-galois}
	Let $L/M/K$ be a tower of fields. Assume $L/K$ and $M/K$ are Galois.
	Then we obtain a short exact sequence
	$$
	1 \to \text{Gal}(L/M) \to \text{Gal}(L/K) \to \text{Gal}(M/K) \to 1
	$$
	of profinite topological groups.
\end{lemma}


\bibliographystyle{plain}
\bibliography{ref}